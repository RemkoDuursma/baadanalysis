

\documentclass[a4paper]{article}

\usepackage{amssymb,amsmath}
\usepackage{palatino}
\usepackage{framed}
\usepackage{parskip}
\usepackage{geometry}
\geometry{verbose,tmargin=3cm,bmargin=3cm,lmargin=3cm,rmargin=3cm}

\begin{document}



\title{Biomass allocation, plant functional types and leaf area vs. mass}

\author{Remko Duursma, Daniel Falster and BAAD data contributors}

\maketitle


%--------------------------------------------------------------------------------------------%

\section{Main text}

\begin{enumerate}
  \item Importance of biomass allocation in earth system models, feedback on fluxes of CO2, lifetime of C in the system.
  \item Lots of approaches to model biomass allocation, these models tend to settle on very simple algorithms.
  \item Little is known about differences between plant functional types, which are accounted for in many ways in these models already, especially via physiology etc. etc. 
  \item We now have a dataset that we can use to evaluate differences between plant functional types.
  \item Important constraints on allocation are a) leaf mass and leaf area fraction, b) root-shoot ratio and c) leaf mass and area per unit sapwood area.
  \item Many models allocate some percentage of NPP to each biomass pool, and then multiply this by SLA to get leaf area, which is the area relevant for fluxes of CO2, H2O and light interception.
  \item Because PFTs differ so much in SLA, and little is known about fNPP to each pool by pft, this would lead to \emph{certain pft differences}.
  \item Here we show that, in fact, it is far more likely that trees vary allocation to end up with a given leaf area / sapwood area ratio, or equivalently, a given leaf area ratio. These fractions, when expressed as mass fraction were far more variable between pft's.
  \item These results major implications evaluation and development of allocation models employed in earth system models.
\end{enumerate}



\section{Methods}



\end{document}
