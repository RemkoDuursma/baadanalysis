

\documentclass[a4paper]{article}

\usepackage{amssymb,amsmath}
\usepackage{palatino}
\usepackage{framed}
\usepackage{parskip}
\usepackage{geometry}
\geometry{verbose,tmargin=3cm,bmargin=3cm,lmargin=3cm,rmargin=3cm}

\begin{document}



\title{Biomass allocation, plant functional types and leaf area vs. mass}

\author{Remko Duursma, Daniel Falster and BAAD data contributors}

\maketitle


%--------------------------------------------------------------------------------------------%

\section{Main text}

\begin{enumerate}
  \item Importance of biomass allocation in earth system models, feedback on fluxes of CO2, lifetime of C in the system.
  \item Lots of approaches to model biomass allocation, these models tend to settle on very simple algorithms.
  \item Little is known about differences between plant functional types, which are accounted for in many ways in these models already, especially via physiology etc. etc. 
  \item We now have a dataset that we can use to evaluate differences between plant functional types.
  \item Important constraints on allocation are a) leaf mass and leaf area fraction, b) root-shoot ratio and c) leaf mass and area per unit sapwood area.
  \item Many models allocate some percentage of NPP to each biomass pool, and then multiply this by SLA to get leaf area, which is the area relevant for fluxes of CO2, H2O and light interception.
  \item Because PFTs differ so much in SLA, and little is known about fNPP to each pool by pft, this would lead to \emph{certain pft differences}.
  \item Here we show that, in fact, it is far more likely that trees vary allocation to end up with a given leaf area / sapwood area ratio, or equivalently, a given leaf area ratio. These fractions, when expressed as mass fraction were far more variable between pft's.
  \item These results major implications evaluation and development of allocation models employed in earth system models.
\end{enumerate}


Friend et al. (2014) showed that 30\% of the uncertainty in global vegetation carbon simulated with seven vegetation models is due to the residence time of carbon in vegetation. This residence time depends on turnover rates of the various plant pools, and hence also on the distribution of biomass between these pools. For example, vegetation with a higher leaf mass fraction will have a lower overall residence time of carbon.

De Kauwe et al. (2014) : "We have shown that allocation approaches that are constrained by biomass fractions (such as functional relationships) were more successful at capturing observed trends, and were generally more robust, than approaches based on allocation coefficients."

Poorter et al. (2012) found a slightly larger LMF in gymnosperms, but the difference was small and could not split between deciduous and evergreen angiosperms due to a lack of data. They argue that "most conifers retain leaves for two or more years. Thus, the higher LMF may simply be the consequence of leaf longevity rather than higher assimilate partitioning per se. In this way, conifers are able to compensate for their lower physiological activity per gram of leaf." This raises the question whether LMF differs by angiosperms and gymnosperms, or as a consequence of leaf longevity (especially deciduous vs. evergreen).

Serious lack of biomass data to constrain models. Studies like Wolf et al. (2011) resort to studying individual tree allometry based on dividing stand biomass by stand density. Wolf et al. (2010) did not find any differences in biomass fractions (root,stem,leaf) between angiosperms and gymnosperms (but again this was based on stand biomass data, but still this is strange!).


\section{Methods}



\end{document}
