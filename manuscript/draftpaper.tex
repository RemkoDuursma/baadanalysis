

\documentclass[a4paper]{article}


\usepackage{amssymb,amsmath}
\usepackage{palatino}
\usepackage{framed}
\usepackage{parskip}
\usepackage{geometry}
\geometry{verbose,tmargin=3cm,bmargin=3cm,lmargin=3cm,rmargin=3cm}
\bibliographystyle{plain}

\begin{document}



\title{Specific leaf area explains difference in biomass partitioning between woody plant functional types}

\author{Remko Duursma, Daniel Falster and BAAD data contributors}

\maketitle


%--------------------------------------------------------------------------------------------%

\section{Main text}

\begin{enumerate}
\item Importance of biomass allocation in global vegetation models \cite{ise_comparison_2010}, feedback on fluxes of CO2, lifetime of C in the system \cite{friend_carbon_2014}.
  
\item There is no shortage of theoretical developments in understanding biomass allocation (\cite{cannell_carbon_1994,franklin_modeling_2012}, WBE). In particular, theoretical work focusing on optimal partitioning theory is promising (e.g. \cite{valentine_modeling_2012}) but is yet to lead to operational models for dynamic global vegetation models.
  
\item Previous syntheses on plant biomass relationships has predominantly focused on the universal nature of size-biomass relationships in plants. Little is known about differences between plant functional types (evergreen vs. deciduous, angiosperm vs. gymnosperm). Since most global models parameterize allocation submodels by PFT, it is imperative that we better understand these potential differences.
  
  
\item Whichever is the 'right' allocation model, what is lacking is a solid empirical base that can be used to constrain any model of allocation, that includes woody plants of all sizes, of all major plant functional types and biomes.
  
\item Here we use such a database (Falster et al.) to evaluate whether important differences exist in biomass partitioning woody plant functional types, and whether climatic differences exist.
  
\item As a first approximation, which is useful for large scale DGVMs, models should aim to keep leaf area to stem cross-sectional area constant, and vary biomass allocation accordingly. Specific leaf area then gives the ratio of leaf biomass to stem biomass. 


\end{enumerate}






\section{Lit notes}

\cite{friend_carbon_2014} showed that 30\% of the uncertainty in global vegetation carbon simulated with seven vegetation models is due to the residence time of carbon in vegetation. This residence time depends on turnover rates of the various plant pools, and hence also on the distribution of biomass between these pools. For example, vegetation with a higher leaf mass fraction will have a lower overall residence time of carbon.

\cite{de2014does} "We have shown that allocation approaches that are constrained by biomass fractions (such as functional relationships) were more successful at capturing observed trends, and were generally more robust, than approaches based on allocation coefficients."

\cite{friedlingstein_toward_1999} make no mention of plant functional type differences or the need to vary allocation by pft.

\cite{poorter_biomass_2012} found a slightly larger LMF in gymnosperms, but the difference was small and could not split between deciduous and evergreen angiosperms due to a lack of data. They argue that "most conifers retain leaves for two or more years. Thus, the higher LMF may simply be the consequence of leaf longevity rather than higher assimilate partitioning per se. In this way, conifers are able to compensate for their lower physiological activity per gram of leaf." This raises the question whether LMF differs by angiosperms and gymnosperms, or as a consequence of leaf longevity (especially deciduous vs. evergreen).

\cite{enquist_global_2002} report 2.6 times higher leaf mass at given stem mass for conifers compared to angiosperms, but did not separate further, nor interpreted this difference in terms of SLA. Their interpretation is 'that conifers typically retain three cohorts of leaves that have less well-developed [...] mesophyll'. Not sure how that latter bit makes sense, and leaf lifespan interpretation is no good; because it still begs the question why would conifers not simply have 1/3 less leaf mass per cohort? 

\cite{mccarthy_consistency_2007} further demonstrate large difference in LMF between angiosperms and conifers, and note that a large chunk of the residuals is correlated with SLA and LL. They interpret this rather vaguely as 'leaf morphology may constrain leaf mass distribution'. Their SLA was not on the same individuals, but they could have calculated PFT average LAR anyway, but did not.

\cite{oneill_comparative_1981} reported much higher leaf biomass for gymnosperms vs. angiosperms, at a given woody biomass increment. This means (as Niklas reported) a much higher productivity per unit leaf biomass for angiosperms.

Serious lack of biomass data to constrain models. Studies like \cite{wolf_forest_2011} resort to studying individual tree allometry based on dividing stand biomass by stand density. \cite{wolf_allometric_2010} did not find any differences in biomass fractions (root,stem,leaf) between angiosperms and gymnosperms (but again this was based on stand biomass data, but still this is strange!).

\cite{hui_near_2014} Isometric scaling of above vs. belowground biomass in chinese forest biomass database.

\cite{sardans_tree_2013} report similiar LMF for evergreen angio. and gymnosperms, and ca. 40\% less leaf mass for decid. angio. Local study, 3000 plots in Catalonia.

\cite{reich_temperature_2014} "On average, when standardized for MAT and Mstem, conifers tended to have more Mfol and less Mroot than angiosperm forests". Their Fig. 2 shows trends in LMF with MAT etc., in bins of stem size. Also allometric analysis with MAT as covariate, showing significance of MAT. For Angiosperms, our explanation would be that evergreen angio is more common at high MAT. But what about conifers?

\cite{cairns_root_1997} found no difference in root-shoot ratio between gymno and angio.

\cite{delucia_climate-driven_2000} LA/SA decreases with summer VPD for Pinus but not non-Pinus gymno. Also Pinus has much lower LA/SA than non-Pinus; suggesting that we should separate gymnosperms into those groups as well.


\cite{grier_notes:_1974} early study on leaf mass proportionality to sapwood area (and uses mass not area).

\section{Results and Discussion}








\section{Methods}

\begin{itemize}
  \item Removed deciduous gymnosperms, we have a few of those in BAAD (6 species / studyName combinations), but only one includes estimates of leaf area and mass. 
  \item We did not attempt here to include SLA from another database (such as TRY), instead only used SLA directly estimated for the harvested plants.
  \item We used a generalized additive model (GAM) to visualize the relationships. In addition, we used linear mixed-effects models (with species within study as the random effect) to test for significance and to calculate confidence intervals for PFT-specific averages.
\end{itemize}




\bibliography{allometry}

\end{document}
